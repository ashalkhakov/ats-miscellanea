\section*{ЗАКЛЮЧЕНИЕ}
\addtocontents{toc}{Заключение}

В данной работе дипломником был разработан интерпретатор на основе типизированного внутреннего представления первого порядка, где тип объектной программы и типы свободных переменных отражены в типе представления этой программы. Базовое типизированное $\lambda$-исчисление было расширено общей рекурсией, сложением и произведением типов, целыми числами и булевыми переменными, а также соответсвующими этим типам данным деконструкторами: $if$ и $case$, соответственно.

Отличительными чертами интерпретатора являются:
\begin{itemize}
\item сохранение и прогресс: система типов мета-языка гарантирует, что проверка типов для объектного языка корректно предсказывает выполнение программы, а также отсутствие ошибок типов в интерпретаторе во время выполнения объектных программ
\item применимость к открытым программам, то есть программам со свободными переменными: такие переменные должны быть доступны из окружения
\end{itemize}

Приведенный интерпретатор можно расширить другими, более реалистичными особенностями, такими, как определяемые пользователем структуры данных, система ввода-вывода, изменяемые структуры данных и параметрический полиморфизм. Также возможна реализация синтаксического анализатора.