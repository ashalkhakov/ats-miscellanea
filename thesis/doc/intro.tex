\section{Введение}
\label{sec:intro}

В целом, программирование является процессом, подверженным серьезным ошибкам, и на практике получено много доказательств того, что применение системы типов в языке программирования позволяет обнаруживать некоторый класс программных ошибок во время компиляции, до запуска программы.

В данной работе основной интерес проявляется к обеспечению корректности интерпретаторов посредством типов. Такой интерес обусловлен тем, что от интерпретаторов зависит большое количество программ.

Традиционно, интерпретаторы пишут на Си или Си++, и в этом случае сложно получить гарантии (формальной) корректности. Подход, предпринятый в данной работе, состоит в том, чтобы закодировать правила вывода системы типов объектного языка в представлении абстрактного синтаксиса в мета-языке, получив т.н. \emph{типизированное внутреннее представление}.

При разработке интерпретатора первым возникает вопрос о представлении объектного языка в терминах мета-языка. В случае, когда мета-язык является функциональным языком программирования, таким, как OCaml \cite{Remy/appsem} или Haskell \cite{hudak2007history}, обычно определяют (алгебраический) тип данных для представления программ на объектном языке. При этом возникают проблемы, связанные с тем, что вся информация о типах объектного языка теряется в типе его представления. Более того, поддержка переменных, связывания и подстановки требует особого внимания.

В данной работе внутреннее представление основано на типизированном абстрактном синтаксисе первого порядка, где программные переменные заменены индексами де Брауна. \cite{de1972lambda,chen2006implementing} При таком подходе не только тип объектной программы, но и типы свободных переменных объектной программы отражаются в типе ее представления.

Ключевым результатом работы является реализация проверки типов и интерпретатора, использующие типизированное внутреннее представление, которое отражает дерево вывода типов просто-типизированного лямбда-исчисления \cite{church1940formulation} с расширениями, в ATS \cite{Xi:2004:TYPES}.

