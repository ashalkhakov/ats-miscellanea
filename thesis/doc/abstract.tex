\section*{Аңдатпа}
%\ESKDthisStyle{empty}

Интерпретаторлар қазіргі ақпараттанудың негізі болып табылады, соңдықтаң олардың дәлдігінің маңызы жоғары. Ол интерпретатордың кейбір қасиеттіне тәуелді. Кейбір программалық инварианттарды деректердін кезелген түрлерінен бейгелугі болады. Ал орындалуын тексеріп тұруға болады.

Бұл жұмыста интерпретатор қырұға верификация түрлерімен қарапайым тұрдегі лямбда-есептеулер қолданыңған. Дипломник ішні көрсетілім түрлеріне негізделген жұмысшы интерпретаторың құрды. Прогресс және сақтау сияқты есептеуш қасиетті мета-тіл түрлері жүйесіне негезделген және дәрелденген.

Ұсынылған шешемді қолданба-бағытталған тілдерді іске асыруға қолданұға болады. Сон мен бірге тілдік-бағытталған синтаксистік лексикалық талдауштар, атрибуттық грамматикалар сияқты қөлданұға болады.

\newpage
\section*{Аннотация}
%\ESKDthisStyle{empty}

Интерпретаторы являются основой современной информатики, и поэтому их корректность имеет высокую важность. Корректность зависит от некоторых инвариантов. Известно, что некоторые программные инварианты удобно выразить с помощью типов, а их выполнение гарантировать с помощью проверки типов.

В данной работе основанная на типах верификация применяется к построению интерпретатора просто-типизированного лямбда-исчисления. Автором был сконструирован рабочий интерпретатор на основе типизированного внутреннего представления, состоящий из проверки типов и вычислителя. Свойства вычислителя, такие, как прогресс и сохранение, доказаны формально в системе типов мета-языка.

Предлагаемое решение возможно использовать для реализации встраиваемых предметно-ориентированных языков, а также языково-ориентированных инструментов, например, синтаксических и лексических анализаторов, атрибутных грамматик, и т.п.

\newpage
\section*{Abstract}
%\ESKDthisStyle{empty}

Interpreters are at the heart of modern computing environment, and their correctness is of high importance. Correctness depends on a number of invariants, and it is known that certain program invariants can be conveniently captured with types and enforced by the type checker.

In this diploma we are to demonstrate the application of lightweight, type-based verification to the construction of a simply-typed lambda calculus interpreter (with extensions, such as general recursion, products and sums). We were able to integrate a typeful internal representation into a working interpreter, comprised of a type-checker and an evaluator. Progress and type-preservation properties of the evaluator are formally proven in the type system of the meta-language using dependent types.

The proposed solution can be used for the implementation of (embedded) domain-specific languages, and also language-based tools (parsers, lexers, attribute grammars, ORMs, etc.).
